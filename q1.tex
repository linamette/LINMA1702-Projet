\subsection*{Les variables et données du problème   }
Avant toutes choses, nous allons commencer par nommer et détailler les différentes données du problème. Celles-ci sont reprises dans la liste suivante:

\begin{itemize}[before={\renewcommand\makelabel[1]{\makebox[1cm][r]{##1\hspace{.2cm}}}}]
    \item[$T$] le nombre de semaines sur lesquelles on effectue la planification
    \item[$D^t$] la demande de smartphones pour la semaine $t$
    \item[$D_\mathrm{a}$] la durée d'assemblage d'un smartphone, en heures
    \item[$N_\mathrm{o}$] le nombre d'ouvriers
    \item[$N_\mathrm{h}$] le nombre d'heures normales disponibles pour assembler des smartphones, par semaine et par ouvrier
    \item[$N_\mathrm{hs}$] le nombre d'heures supplémentaires disponibles pour assembler des smartphones, par semaine et par ouvrier
    \item[$N_\mathrm{st}$] le nombre maximal de smartphones sous-traitables, par semaine 
    \item[$C_\mathrm{m}$] le coût des matériaux
    \item[$C_\mathrm{s}$] le coût du stockage d'un smartphone
    \item[$C_\mathrm{h}$] le coût horaire d'un ouvrier
    \item[$C_\mathrm{hs}$] le coût d'une heure supplémentaire de la part d'un ouvrier
    \item[$C_\mathrm{st}$] le coût de la sous-traitance pour l'assemblage d'un smartphone
    \item[$C_\mathrm{r}$] le coût supplémentaire associé au retard de livraison d'un smartphone
    \item[$C_\mathrm{e}$] le coût d'embauche d'un ouvrier
    \item[$C_\mathrm{l}$] le coût de licenciement d'un ouvrier
    \item[$S_\mathrm{i}$] le stock initial (et final) de smartphones
\end{itemize}

Viennent ensuite les variables, pour $t \in \{1,\dots,T\}$:

\begin{itemize}[before={\renewcommand\makelabel[1]{\makebox[1cm][r]{##1\hspace{.2cm}}}}]
    \item[$n^t$] le nombre de smartphones produits en conditions normales
    \item[$n_\mathrm{hs}^t$] le nombre de smartphones produits en heures supplémentaires
    \item[$n_\mathrm{st}^t$] le nombre de smartphones produits en sous-traitance
    \item[$s^0$] le stock initial de smartphones
    \item[$s^t$] le stock de smartphones à la fin de la semaine $t$, disponible en semaine $t+1$, pour lequel on paye le stockage en semaine $t$
    \item[$r^t$] les smartphones retardés lors de la semaine $t-1$, à produire et livrer en semaine $t$
\end{itemize}

\subsection*{Les contraintes}

Ceci fait, nous pouvons maintenant lister les contraintes selon lesquelles nous allons optimiser:

\begin{enumerate}
    \item égalité entre
    \begin{itemize}
        \item les smartphones produits et disponible en stock en début de semaine et
        \item la demande, le stock disponible en fin de semaine et le retard de la semaine passé, moins le retard que l'on décide de prendre cette semaine :
    \end{itemize}
    
    \[
        n^t + n_\mathrm{hs}^t + n_\mathrm{st}^t + s^{t-1} = D^t + s^t + r^t - r^{t+1}
        \qquad\quad \forall t \in \{1,\dots,T-1\}
        \text;
    \]
    
    \item afin d'éviter un retard de deux semaines (possible avec la contrainte précédente), le retard doit être inférieur à la demande de la semaine précédente :
    \[
        D^t \geq r^{t+1}
        \qquad\quad \forall t \in \{1,\dots,T-1\}
        \text;
    \]
    
    \item limite sur la capacité en heures normales, supplémentaires et en sous-traitance :
    \[
        n^t \leq \frac{N_\mathrm{h}\,N_\mathrm{o}}{D_\mathrm{a}}
        \text,\qquad
        n_\mathrm{hs}^t \leq \frac{N_\mathrm{hs}\,N_\mathrm{o}}{D_\mathrm{a}}
        \quad\text{et}\quad
        n_\mathrm{st}^t \leq N_\mathrm{st}
        \qquad\quad \forall t \in \{1,\dots,T\}
        \text;
    \]
    
    \item stock initial et final fixé à $S_\mathrm{i}$ :
    \[
        s^0 = S_\mathrm{i} \quad\text{et}\quad s^T = S_\mathrm{i}
        \text;
    \]
    
    \item pas de retard à rattrapper en semaine 1 :
    \[
        r^1 = 0
        \text{;}
    \]
    
    \item enfin, toutes les variables positives :
    \[
        n^t, n_\mathrm{hs}^t, n_\mathrm{st}^t, s^t, s^T, r^t, r^T \geq 0
        \qquad\quad \forall t \in \{1,\dots,T\}
        \text.
    \]
\end{enumerate}

\subsection*{La fonction objectif}

Terminons par la fonction objectif. Soit
$f(x) : \mathbb{R}^{5T+1} \rightarrow \mathbb{R}$,
avec $x$ un vecteur de $\mathbb{R}^{5T+1}$:
\[
    x = (n^1,\dots,n^T,n_\mathrm{hs}^1,\dots,n_\mathrm{hs}^T,n_\mathrm{st}^1,\dots,n_\mathrm{st}^T,s^0,\dots,s^T,r^1,\dots,r^T)
    \text.
\]

Par simplicité, nous allons décomposer $f(x)$ en une somme de fonctions $f^t$ définies pour chaque semaine $t \in \{1,\dots,T\}$:
\[
    f(x) = \sum_{t=1}^{T} f^t(n^t, n_\mathrm{hs}^t, n_\mathrm{st}^t, s^t, r^t)
\]
et
\begin{align*}
    f^t(n^t, n_\mathrm{hs}^t, n_\mathrm{st}^t, s^t, r^t)
    &= (n^t + n_\mathrm{hs}^t)\,C_\mathrm{m} 
    & \text{prix des matériaux}\\
    &+ n_\mathrm{hs}^t\,D_\mathrm{a}\,C_\mathrm{hs}
    & \text{production en heures supplémentaires}\\
    &+ n_\mathrm{st}^t\,C_\mathrm{st}
    & \text{production sous-traitée}\\
    &+ s^t\,C_\mathrm{s}
    & \text{stockage des smartphones}\\
    &+ r^t\,C_\mathrm{r}
    & \text{retard de livraison}
    \text.
\end{align*}

Il est à noter que le salaire en heure normale des ouvriers n'apparaît pas dans cette fonction: nous ne sommes plus au XIXe siècle, un ouvrier est donc toujours payé pour sa semaine, peu importe le travail qu'il a réellement effectué. Il s'agit donc d'un coût fixe que l'on ne peut pas optimiser sans licencier des ouvriers.

\subsection*{Le modèle complet}

En résumé, le problème se modélise de la manière suivante :
\begin{align*}
    \min_{x}\;f(x) =& \sum_{t=1}^{T}
    (n^t + n_\mathrm{hs}^t)\,C_\mathrm{m} 
    + n_\mathrm{hs}^t\,D_\mathrm{a}\,C_\mathrm{hs} \\
    &+ n_\mathrm{st}^t\,C_\mathrm{st}
    + s^t\,C_\mathrm{s}
    + r^t\,C_\mathrm{r} \\[0.5em]
%
    \mathrm{s.c.q.}\;\phantom{f(x) =}
    & n^t + n_\mathrm{hs}^t + n_\mathrm{st}^t + s^{t-1} = D^t + s^t + r^t - r^{t+1} & \forall t \in \{1,\dots,T-1\} \\
    & D^t \geq r^{t+1} & \forall t \in \{1,\dots,T-1\} \\
    & n^t \leq N_\mathrm{h}\,N_\mathrm{o}/D_\mathrm{a} & \forall t \in \{1,\dots,T\} \\
    & n_\mathrm{hs}^t \leq N_\mathrm{hs}\,N_\mathrm{o}/D_\mathrm{a} & \forall t \in \{1,\dots,T\} \\
    & n_\mathrm{st}^t \leq N_\mathrm{st} & \forall t \in \{1,\dots,T\} \\
    & s^0 = S_\mathrm{i} \\
    & s^T = S_\mathrm{i} \\
    & r^1 = 0 \\
    & n^t, n_\mathrm{hs}^t, n_\mathrm{st}^t, s^0, s^t, r^t \geq 0 & \forall t \in \{1,\dots,T\} \\
\end{align*}

À ce stade-ci, on peut se demander si l'hypothèse des variables non-entières n'entraînera pas des résultats aberrants, tel qu'un nombre de smartphones produits non-entier.
