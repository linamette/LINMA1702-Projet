À modifier par rapport à q1 :

- Ajouter une nouvelle série de variables n_o^t, n_e^t, n_l^t, càd le nombre
  d'ouvriers, d'embauches et de licenciements en début de semaine t (on
  considère n *après* embauche et licenciement)

- Ajouter une contrainte
    n_o^t, n_e^t, n_l^t >= 0
  qui garantit la positivité des nouvelles variables.

- Ajouter une contrainte
    n_o^t - n_o^{t-1} = n_e^t - n_l^t
  qui "garde en mémoire" l'évolution du nombre d'ouvriers.

- Modifier la contrainte
    n_hs^t <= N_{hs,max}
  pour prendre en compte que N_{hs,max} n'est plus constant, mais dépend
  maintenant linéairement de n_o^t.

- Ajouter le salaire à la fonction objectif : on se retrouve avec
    f(...) = ... + n_o*C_o
  avec C_o le salaire hebdomadaire d'un ouvrier.
