Afin d'inclure la variation du nombre d'employés, nous introduisons trois nouvelles variables pour chaque semaine: $n_\text{o}^t$, le nombre d'ouvriers après embauche et licenciement, $n_\text{e}^t$ le nombre d'embauches et $n_\text{l}^t$, le nombre de licenciements.

De nouvelles contraintes sur ces variables sont ajoutées:
\begin{enumerate}
    \item nouvelles variables positives:
    \[
        n_\text{o}^t, n_\text{e}^t, n_\text{l}^t \geq 0
        \text;
    \]
    
    \item la variation du nombre d'ouvrier doit dépendre du nombre d'embauche et de licenciement:
    \[
        n_\text{o}^t - n_\text{o}^{t-1} = n_\text{e}^t - n_\text{l}^t
        \text;
    \]
    
    \item les limites sur la capacité en heures normales et supplémentaires changent pour dépendre du nombre d'ouvriers :
    \[
        n_hs^t \leq N_{hs,max}
        \text.
    \]
\end{enumerate}

Enfin, la fonction-objectif est modifiée pour inclure:
\begin{itemize}
    \item le salaire des ouvriers, qui n'est plus constant:
    \[
        n_\text{o}^t*C_\text{o}
        \text{ et }
    \]
    
    \item le prix d'embauche et de licenciement:
    \[
        n_\text{e}^t*C_\text{e} + n_\text{l}^t*C_\text{l}
        \text.
    \]
\end{itemize}
