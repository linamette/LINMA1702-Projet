Afin d'inclure la variation du nombre d'employés, nous introduisons trois nouvelles variables pour chaque semaine $t$:

\begin{itemize}[before={\renewcommand\makelabel[1]{\makebox[1cm][r]{##1\hspace{.2cm}}}}]
    \item[$n_\mathrm{o}^t$] le nombre d'ouvriers après embauche et licenciement
    \item[$n_\mathrm{e}^t$] le nombre d'embauches
    \item[$n_\mathrm{l}^t$] le nombre de licenciements.
\end{itemize}

De nouvelles contraintes sur ces variables sont ajoutées:

\begin{itemize}
    \item nouvelles variables positives:
    \[
        n_\mathrm{o}^t, n_\mathrm{e}^t, n_\mathrm{l}^t \geq 0
        \text;
    \]
    
    \item la variation du nombre d'ouvrier doit dépendre du nombre d'embauche et de licenciement:
    \[
        n_\mathrm{o}^t - n_\mathrm{o}^{t-1} = n_\mathrm{e}^t - n_\mathrm{l}^t
        \text;
    \]
    
    \item les limites sur la capacité en heures normales et supplémentaires changent pour dépendre du nombre d'ouvriers :
    \[
        n^t \leq N_\mathrm{h}\,n_\mathrm{o}^t/D_\mathrm{a}
        \quad\mathrm{et}\quad
        n_\mathrm{hs}^t \leq N_\mathrm{hs}\,n_\mathrm{o}^t/D_\mathrm{a}
        \text.
    \]
\end{itemize}

Enfin, la fonction-objectif est modifiée pour inclure:

\begin{itemize}
    \item le salaire des ouvriers, qui n'est plus constant:
    \[
        f^t(\dots,n_\mathrm{o}^t)
        = \dots + n_\mathrm{o}^t\,C_\mathrm{h}
        \text;
    \]
    
    \item le prix d'embauche et de licenciement:
    \[
        f^t(\dots,n_\mathrm{e}^t,n_\mathrm{l}^t)
        = \dots + n_\mathrm{e}^t\,C_\mathrm{e} + n_\mathrm{l}^t\,C_\mathrm{l}
        \text.
    \]
\end{itemize}
