Plusieurs critiques peuvent être émises sur la modélisation.

Tout d'abord, parlons de  l'embauche. Bien que le modèle prenne en compte un coût d'embauche pour chaque nouvel employé, il oublie le fait qu'un nouveau ouvrier travaille moins bien qu'un ouvrier expérimenté. Il prendra, à priori, plus de temps pour assembler un smartphone. Afin de combler ce manque, il serait utile d'ajouté un type de variables, le nombre de nouveaux ouvriers engagés lors d'une semaine, et un paramètre, la durée d'assemblage pour des nouveaux ouvriers . Ainsi, une personne engagée en semaine \textit{T}, sera durant un certain nombre de semaines \textit{Nb} capable de produire des produits à une durée d'assemblage \textit{D-a-nouveau} et elle sera ajoutée au nombre de nouveau employés. Le nombre d'ouvriers restera constant. Après \textit{Nb} semaines, l'ouvrier est placé dans les ouvriers classiques. Le nombre de nouveaux ouvriers diminue de 1 et celui des ouvriers normaux augmente de 1.

Le modèle ne prend pas en compte les jours fériés légaux qui peuvent survenir pendant les semaines sur lesquelles est établie l'optimisation. Ceci  amène une perte de précision et une diminution du coût réel. En effet, si les ouvriers ne peuvent travailler un jour férié légal, il sera peut-être obligatoire d'engager plus de personnel pour répondre à la demande ou de sous-traiter une plus grande quantité de produit. Pour pallier ce problème, il faudrait passer en paramètre le nombre de jour férié présent dans chacune des semaines de planification. Notons néanmoins que si l'usine est localisée en Chine, ce problème ne se posera jamais.