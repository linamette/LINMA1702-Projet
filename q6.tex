Pour introduire un coût variable des heures supplémentaires, il nous faut ajouter $N_\mathrm{hs}$ variables par semaine $t$: $\xi_0^t, \xi_1^t, \dots, \xi_{N_\mathrm{hs}-1}^t$. Celles-ci sont toutes comprises entre $0$ et $1$, et elles représentent quelle proportion de chaque heure supplémentaire a été effectuée, pour tous les ouvriers.
Par exemple, si les 10 ouvriers ont effectué ensemble 17 heures supplémentaires à la semaine $t$, on aura $\xi_0^t = 1$, $\xi_1^t = 0.7$ et $\xi_i^t = 0$ pour $i \geq 2$.

Une seule contrainte est nécessaire pour ces variables, il s'agit des bornes supérieures et inférieures:
\[
    \xi_i^t \in [0, 1]
    \quad\forall i
    \text.
\]

Enfin, il est nécessaire de modifier la partie \og{}production en heures supplémentaires\fg{} de la fonction-objectif pour prendre en compte le coût variable:
\[
    f^t(\dots, \xi_0^t, \xi_1^t, \dots, \xi_{N_\mathrm{hs}-1}^t)
    = \dots + C_\mathrm{hs}\;N_\mathrm{o}(\xi_0^t + \xi_1^t \alpha^1 + \dots + \xi_{N_\mathrm{hs}-1}^t \alpha^{N_\mathrm{hs}-1})
    \text,
\]
avec $\alpha$ le coefficient d'augmentation du coût des heures supplémentaire. Dans l'exemple de l'énoncé, $\alpha = 1.05$.

Étant donné que le coût de l'heure $i$ est plus important que celui de l'heure $i-1$ d'un facteur $\alpha$, l'optimisation mettra toujours les variables $\xi$ à 1 dans l'ordre. Ceci explique qu'il n'est pas nécessaire d'imposer une contrainte sur l'ordre des $\xi$.

Cependant, cela n'est le cas que pour $\alpha \geq 1$: en effet, si le coût diminuait, il serait plus profitable de mettre d'abord les derniers $\xi$ à 1, car ceux-ci correspondent aux dernières heures supplémentaires, qui sont les moins chères. Cela reviendrait à travailler d'abord les dernières heures, puis éventuellement de travailler les plus récentes, ce qui n'a pas de sens.

Pour résoudre ce problème, on serait tenté d'ajouter une contrainte du type:
\[
    \xi_i^t \geq \xi_{i+1}^t
    \text,
\]
mais les variables $\xi$ n'étant pas entières, cela n'empêcherait pas d'avoir un résultat où toutes $\xi_i^t$ a une certaine valeur inférieure à 1 pour tout $i$. Par exemple, si l'on veut effectuer 10 heures supplémentaires par ouvrier et que $N_\mathrm{hs} = 5$, on obtiendra un optimum pour $\xi_i^t = 0.2 \;\forall i$.

En fait, il n'est pas possible de trouver une contrainte linéaire qui permette d'imposer l'ordre des variables $\xi$, pour la simple raison que l'ensemble $\{(\xi_i, \xi_{i+1}) \in [0, 1]\!\times\![0, 1] \,|\, \xi_{i+1} = 0 \text{ si } \xi_i \neq 0\}$ est l'union de deux segments de droite, et un tel ensemble n'est pas convexe (voir fig. \ref{fig:nonconvexe}).

\begin{figure}[h] \label{fig:nonconvexe}
    \centering
    \setlength{\unitlength}{0.25cm}
    \begin{picture}(12,12)
    \thinlines
    \put(0,0){\vector(0,1){12}}
    \put(0,0){\vector(1,0){12}}
    
    \put(-3,11){$\xi_{i+1}$}
    \put(11,-1.5){$\xi_i$}
    
    \linethickness{.8mm}
    \put(0,0){\line(1,0){10}}
    \put(10,0){\line(0,1){10}}
    
    \thicklines
    \color{red}
    \put(2,0){\line(3,2){8}}
    
    \end{picture}
    \caption{Il n'est pas possible de tracer lier un point du segment du dessous à un point du segment de droite tout en restant dans l'ensemble formé par ces deux segments. L'ensemble est donc convexe. }
\end{figure}

