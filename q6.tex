Le coût variable pour les heures supplémentaires modifie radicalement le problème, le rendant non-linéaire. En effet, si on considère $C$ le coût d'une heure supplémentaire et $n$ le nombre d'heures supplémentaires, avec un taux d'augmentation de $5\%$, le coût total des heures supplémentaires $C_\text{tot}$ est le suivant:
\begin{align*}
    C_\text{tot}
    &= C + C (105\%) + C (105\%)^2 + \dots + C (105\%)^{n-1}
    &= \sum_{i=0}^{n-1} C (105\%)^i
    \text.
\end{align*}
Nous voyons aisément, parce-que $n$ apparaît en exposant, que $C_\text{tot}$ n'est plus linéaire selon cette variable.

Le problème complet se présente donc de la manière suivante :
\[
    \text{min}_{\dots,n,\dots} f(\dots,n,\dots) = \dots + C_\text{tot} + \dots
    \text.
\]
Le reste de la fonction objectif ne changeant pas, nous l'avons omis. Seul $C_\text{tot}$ est différent. Comme celui-ci n'est plus linéaire, $f$ n'est plus non plus linéaire selon l'une des variables par rapports auxquelles on optimise. Le problème n'est donc plus linéaire.

La même situation apparaît si l'on voulait diminuer le coût des heures supplémentaires: il suffit de remplacer $105\%$ par une valeur inférieure à $100\%$. On obtiendra une exponentielle décroissante, qui n'est bien évidemment toujours pas linéaire.
